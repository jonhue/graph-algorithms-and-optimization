% !TeX root = ../main.tex
% Add the above to each chapter to make compiling the PDF easier in some editors.

\chapter{Conductance and Expanders}

\section{Conductance}

\begin{defn}[Volume] The \emph{volume}\index{volume} of a set of vertices $\sS \subseteq \sV$ is the sum of weighted degrees, \begin{align}
    \vol(S) \defeq \sum_{v \in \sS} \vd(v) = \trans{\vOne_\sS}\vd = \trans{\vOne_\sS}\mD\vOne_\sS.
\end{align}
\end{defn}

A \emph{cut}\index{cut} $(\sS, \sV \setminus \sS)$ is a proper subset of vertices, $\emptyset \subset \sS \subset \sV$ partitioning vertices into two sets $\sS$ and $\sV \setminus \sS$.

\begin{defn}[Value of a cut] The \emph{value}\index{cut value} of a cut $(\sS, \sV \setminus \sS)$ is the sum of weights of crossing edges, \begin{align}\begin{split}
    c(\sS) &\defeq \sum_{\substack{\{a,b\} \in \sE \\ a \in \sS,\ b \in \sV \setminus \sS}} \vw(\{a,b\}) \\
    &= \sum_{\{a,b\}\in\sE} \vw(\{a,b\}) [\vOne_\sS(a) - \vOne_\sS(b)]^2 = \trans{\vOne_\sS}\mL\vOne_\sS.
\end{split}\end{align}
\end{defn}

\begin{defn}[Conductance of a cut] The \emph{conductance}\index{conductance} of a cut $(\sS, \sV \setminus \sS)$ is, \begin{align}
    \phi(\sS) \defeq \frac{c(\sS)}{\min\{\vol(\sS), \vol(\sV \setminus \sS)\}} = \phi(\sV \setminus \sS) \in [0,1],
\end{align} where, if the graph has unit weights, $c(\sS) = |\sE(\sS, \sV \setminus \sS)|$ counts the number of crossing edges.
\end{defn}

\begin{rmk}
If $\vol(\sS) \leq \vol(\sV \setminus \sS)$, then we can write \begin{align}
    \phi(\sS) = \frac{\trans{\vOne_\sS}\mL\vOne_\sS}{\trans{\vOne_\sS}\mD\vOne_\sS}.
\end{align}
\end{rmk}

\begin{defn}[Conductance of a graph]The \emph{conductance} of a graph $G$ is the smallest conductance of all cuts, \begin{align}
    \phi(G) \defeq \min_{\emptyset \subset \sS \subset \sV} \phi(S).
\end{align}
\end{defn} Thus, $\phi(G)$ is small if there is a ``good'' cut (with few crossing edges relative to the volume of the parts). In contrast, if $\phi(G)$ is large, then $G$ is well-connected, i.e., there is no ``good'' cut.

\begin{defn}[Expander and expander decomposition] For any $\phi \in (0,1]$, if $\phi(G) \geq \phi$, then $G$ is called a \emph{$\phi$-expander}\index{expander}.

A \emph{$\phi$-expander decomposition}\index{expander decomposition} of \emph{quality}\index{quality} $q$ is a partition of the vertex set $\sV = \sX_1 \cup \dots \cup \sX_k$ such that \begin{enumerate}
    \item $G[\sX_i]$ is a $\phi$-expander; and
    \item the number of edges not contained in any $G[\sX_i]$ is at most $q \phi m$, i.e. only ``few'' edges cross the parts.
\end{enumerate}
\end{defn}

In \cref{cha:expanders_using_max_flow}, we discuss how we can efficiently find an expander decomposition. Let us consider a few examples.

\begin{exc}\label{exc:expander_complete_graph}
$\phi(K_n) = \frac{n}{2(n-1)}$. So $K_n$ is a $\frac{1}{2}$-expander.
\end{exc}

\begin{exc}\label{exc:expander_path_graph}
$\phi(P_n) = \frac{1}{n-1}$. So $P_n$ is a $\frac{1}{n}$-expander.
\end{exc}

\begin{lem}
If $G$ is a connected $\phi$-expander with unit weights, then we have for the diameter $D$ of $G$, \begin{align}
    D = \LandauO{\frac{\log m}{\phi}}.
\end{align}
\end{lem}
\begin{proof}
Fix any pair of vertices $s,t \in \sV$. Let $B(s,d)$ be the closed ball around $s$ of radius $d$. Let $E(B(s,d))$ be the internal edge set of $B(s,d)$.

Observe that since $G$ is connected, we have $|E(B(s,0))| \geq 1$. Moreover, for each $d \geq 0$ where $|E(B(s,d))| \leq \frac{m}{2}$, we have by the definition of a $\phi$-expander that \begin{align*}
    |E(B(s,d), \sV \setminus B(s,d))| \geq \phi \cdot |E(B(s,d))|.
\end{align*} Thus, $|E(B(s,d+1))| \geq (1+\phi)|E(B(s,d))|$. Let $d_{\max}$ be the largest integer such that $|E(B(s,d_{\max}))| \leq \nicefrac{m}{2}$. We have $d_{\max} \leq \nicefrac{2\log m}{\phi}$ as otherwise, \begin{align*}
    |E(B(s,d_{\max}))| > (1+\phi)^{\nicefrac{2 \log m}{\phi}} \geq (1 + \nicefrac{\phi}{2} + (\nicefrac{\phi}{2})^2)^{\nicefrac{2 \log m}{\phi}} \geq e^{\log m} = m, \margintag{using $e^x < 1 + x + x^2$ for $x < 1.79$}
\end{align*} which gives a contradiction to the fact that the number of edges in $G$ is $m$. Thus, for a radius of $\nicefrac{2 \log m}{\phi} + 1$, the ball centered at $s$ has more than $\nicefrac{m}{2}$ edges.

Finally, follow the same argument from $t$. As both balls contain more than $\nicefrac{m}{2}$ edges, they must intersect in at least one edge. But this implies that there is a $s$-$t$ path of length $\LandauO{\nicefrac{\log m}{\phi}}$.
\end{proof}

% \begin{thm}
% There is a procedure \textsc{CertifyOrCut($G, \phi$)} that either \begin{itemize}
%     \item certifies that $G$ is a $\phi$-expander; or
%     \item presents a cut $\sS$ such that $\phi(S) \leq \sqrt{2\phi}$.\footnote{Finding the best cut (i.e., a cut with conductance $\phi$) is NP-hard.}
% \end{itemize}
% \end{thm}
% \begin{proof}
% TBD
% \end{proof}

% \begin{thm}
% There is an algorithm that computes a $\phi$-expander decomposition of quality $q = \LandauO{\phi^{-\nicefrac{1}{2}} \log n}$ in time $\LandauO{m \log^c n}$ for some constant $c$.
% \end{thm}
% \begin{proof}
% TBD
% \end{proof}

\section{Cheeger's Inequality}

\begin{thm}[Cheeger's inequality]\index{Cheeger's inequality} We have for a graph $G$ and its normalized Laplacian matrix $\mN$, \begin{align}
    \frac{\lambda_2(\mN)}{2} \leq \phi(G) \leq \sqrt{2 \lambda_2(\mN)}.
\end{align}
\end{thm}\noindent In words, $\lambda_2(\mN)$ approximates the conductance $\phi(G)$ up to a square root. This is why we say that $\lambda_2(\mN)$ is a measure of connectivity of a graph.

\begin{proof}
TBD
\end{proof}

\section{Sparsity}

A concept related to conductance is the notion of \emph{sparsity}.

\begin{defn}[Sparsity] The \emph{sparsity}\index{sparsity} of a cut $(\sS, \sV \setminus \sS)$ is, \begin{align}
    \sigma(\sS) \defeq \frac{c(\sS)}{\min\{|\sS|, |\sV \setminus \sS|\}} = \sigma(\sV \setminus \sS) \in [0, \max_{v \in \sV} \vd(v)].
\end{align} The sparsity of a graph is again defined as the smallest sparsity of all cuts.
\end{defn}
\begin{rmk}
If $|\sS| \leq |\sV \setminus \sS|$, then we can write \begin{align}
    \sigma(\sS) = \frac{\trans{\vOne_\sS}\mL\vOne_\sS}{\trans{\vOne_\sS}\vOne_\sS}.
\end{align}
\end{rmk}

\begin{lem}
We have for any cut $(\sS, \sV \setminus \sS)$ in a connected unit weight graph that $\sigma(\sS) \geq \phi(\sS)$.
\end{lem}
\begin{proof}
As the graph is connected and has unit weights, $\vol(\sS) = \sum_{v \in \sS} \vd(v) \geq |\sS|$.
\end{proof}

An alternative version of Cheeger's inequality relates the second eigenvalue of $\mL$ (not $\mN$!) to the sparsity of the graph.

\begin{fct}[Cheeger's inequality for sparsity] We have for a graph $G$ and its Laplacian matrix $\mL$, \begin{align}
    \frac{\lambda_2(\mL)}{2} \leq \sigma(G) \leq \sqrt{2 \lambda_2(\mL) \max_{v \in \sV} \vd(v)}.
\end{align}
\end{fct}