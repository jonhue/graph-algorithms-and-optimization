% !TeX root = ../main.tex
% Add the above to each chapter to make compiling the PDF easier in some editors.

\chapter{Introduction to Spectral Graph Theory}

Spectral graph theory studies graphs through linear algebra. The fundamental object that we will work with is the Laplacian matrix that we introduced in \cref{defn:laplacian_matrix}.

\section{Eigenvalues of the Laplacian Matrix}

\begin{lem}
Denote by $\sX_1 \cup \dots \cup \sX_k = V$ the connected components of a graph $G$. Then we have for the Laplacian matrix $\mL$ of $G$, \begin{align}
    \ker{\mL} = \vspan\{\vOne_{\sX_1}, \dots, \vOne_{\sX_k}\},
\end{align} where $\vOne_{\sX}(v) = \Ind{v \in \sX}$. In particular, $\ker{\mL} = \vspan\{\vOne\}$ if $G$ is connected.
\end{lem}
\begin{proof}
TBD
\end{proof}
\begin{cor}
When $G$ has $k$ connected components, then $\lambda_i(\mL) = 0$ for all $i \in [k]$.
\end{cor}

In the following, we will study connected graphs. If a graph consists of multiple connected components, we may study each individually.

\begin{lem}
$\mL \preceq 2\mD$.
\end{lem}
\begin{proof} It can be shown that $\mD + \mA \succeq \mZero$,\footnote{The proof is similar to the proof that $\mL = \mD - \mA \succeq \mZero$. $\mD + \mA$ is also called the \emph{signless Laplacian matrix}\index{signless Laplacian matrix} and is an interesting object in its own right: its eigenvalues can be used to count edges and identify bipartite components.} so, $\mD \succeq -\mA$, and hence, $\mL \preceq 2\mD$.
\end{proof}

\subsection{Electrical Energy}

\begin{lem}
For any voltages $\vx \in \R^{|\sV|}$, \begin{align}
    \lambda_2(\mL) \norm{\vx}_2^2 \leq \spa{E}(\vx) \leq \lambda_n(\mL) \norm{\vx}_2^2.
\end{align}
\end{lem}
\begin{proof}
TBD
\end{proof}

\begin{lem}
Voltages $\vx \in \R^{|\sV|}$ routing demands $\vd \perp \vOne$ satisfy, \begin{align}
    \frac{\norm{\vd}_2^2}{\lambda_n(\mL)} \leq \spa{E}(\vx) \leq \frac{\norm{\vd}_2^2}{\lambda_2(\mL)}.
\end{align}
\end{lem}
\begin{proof}
Recall from \cref{eq:electrical_energy_demands} that $\mathcal{E}(\vx) = \trans{\vd}\pinv{\mL}\vd$. Using Courant-Fischer, \begin{align*}
    \lambda_n(\pinv{\mL}) &= \max_{\vx \neq \vZero} \frac{\trans{\vx}\pinv{\mL}\vx}{\norm{\vx}_2^2} \geq \frac{\trans{\vd}\pinv{\mL}\vd}{\norm{\vd}_2^2} \\
    \lambda_2(\pinv{\mL}) &= \min_{\substack{\vx \neq \vZero \\ \vx \perp \vOne}} \frac{\trans{\vx}\pinv{\mL}\vx}{\norm{\vx}_2^2} \leq \frac{\trans{\vd}\pinv{\mL}\vd}{\norm{\vd}_2^2}, \margintag{using that $\ker{\pinv{\mL}} = \ker{\mL} = \vspan\{\vOne\}$}
\end{align*} using that $\vd \neq \vZero$ and $\vd \perp \vOne$. Rearranging the inequalities, we obtain, \begin{align*}
    \lambda_2(\pinv{\mL}) \norm{\vd}_2^2 \leq \trans{\vd}\pinv{\mL}\vd \leq \lambda_n(\pinv{\mL}) \norm{\vd}_2^2.
\end{align*} Finally, recall from \cref{eq:pseudoinverse_spectrum} that $\lambda_2(\pinv{\mL}) = \inv{\lambda_n(\mL)}$ and $\lambda_n(\pinv{\mL}) = \inv{\lambda_2(\mL)}$.
\end{proof}

\subsection{Useful Inequalities}

\begin{lem}[Path inequality]\index{path inequality} We have that \begin{align}
    (n-1) P_n \succeq G_{1,n},
\end{align} where $P_n$ is the unit weight path graph on $n$ vertices and $G_{i,j}$ is the unit weight graph on $n$ vertices with the single edge $\{i,j\}$.
\end{lem}
\begin{proof}
Fix any $\vx \in \R^n$ and let $\vDelta(i) \defeq \vx(i+1) - \vx(i)$. We have, \begin{align*}
    \trans{\vx}G_{1,n}\vx &= (\vx(n) - \vx(1))^2 \\
    &= \parentheses*{\sum_{i=1}^{n-1} \vDelta(i)}^2 \\
    &= (\trans{\vOne_{n-1}}\vDelta)^2 \\
    &\leq \norm{\vOne_{n-1}}_2^2 \norm{\vDelta}_2^2 \margintag{using Cauchy-Schwarz} \\
    &= (n-1) \sum_{i=1}^{n-1} \vDelta(i)^2 \\
    &= (n-1) \sum_{i=1}^{n-1} (\vx(i+1) - \vx(i))^2 \\
    &= (n-1)\trans{\vx}P_n\vx. \qedhere
\end{align*}
\end{proof}

\begin{lem}
For any unit weight graph $G$ on $n$ vertices with diameter\footnote{The \emph{diameter}\index{diameter} of a graph is the length of its longest path.} $D$, \begin{align}
    \lambda_2(G) \geq \frac{1}{nD}.
\end{align}
\end{lem}
\begin{proof} We denote by $G^{i,j}$ the subgraph of $G$ consisting of the shortest $i-j$ path. We have, \begin{align*}
    K_n = \sum_{i < j} G_{i,j} &\preceq \sum_{i<j} \underbrace{(j-i)}_{\leq D} \underbrace{G^{i,j}}_{\subseteq G} \margintag{analogously to the path inequality} \\
    &\preceq n^2 D G.
\end{align*} Thus, $n^2 D \lambda_2(G) \geq \lambda_2(K_n) = n$, and hence, $\lambda_2(G) \geq \frac{1}{nD}$.
\end{proof}

\section{Examples}

\begin{lem}[Spectrum of the complete graph] \begin{align}
    \lambda_2(K_n) = \dots = \lambda_n(K_n) = n.
\end{align}
\end{lem}
\begin{proof} We have, $\mA = \vOne\trans{\vOne} - \mI$ and $\mD = (n-1)\mI$, so $\mL = n\mI - \vOne\trans{\vOne}$. For any $\vx \perp \vOne$, $\mL\vx = n\vx$.
\end{proof}

We now want to better understand $\lambda_2$ and $\lambda_n$ for some common graphs. The tools we will use are the following: \begin{itemize}
    \item \underline{to lower bound $\lambda_2(G)$:} Relate the eigenvalues of $K_n$ and $G$, yielding $K_n \preceq f(n) G$. Knowing that $\lambda_2(K_n) = n$, we have, $f(n) \lambda_2(G) \geq n$, and hence, $\lambda_2(G) \geq \frac{n}{f(n)}$.
    \item \underline{to upper bound $\lambda_2(G)$:} Due to Courant-Fischer, \begin{align*}
        \lambda_2(G) = \min_{\substack{\vx \perp \vOne \\ \vx \neq \vZero}} \frac{\trans{\vx}\mL\vx}{\trans{\vx}\vx} \leq \frac{\trans{\vy}\mL\vy}{\trans{\vy}\vy},
    \end{align*} for any $\vy \perp \vOne, \vy \neq \vZero$. We can therefore find a so-called \emph{test vector}\index{test vector} $\vy$ with these properties.
    \item \underline{to lower bound $\lambda_n(G)$:} Similarly, due to Courant-Fischer, \begin{align*}
        \lambda_n(G) = \max_{\vx \neq \vZero} \frac{\trans{\vx}\mL\vx}{\trans{\vx}\vx} \geq \frac{\trans{\vy}\mL\vy}{\trans{\vy}\vy},
    \end{align*} for any $\vy \neq \vZero$.
    \item \underline{to upper bound $\lambda_n(G)$:} Using that $\mL \preceq 2 \mD$, we have, $\lambda_n(G) \leq 2 \max_{v \in \sV} \vd(v)$, where $\vd(v)$ is the weighted degree of $v$.
\end{itemize}

\subsection{Path Graph}

\begin{lem}
$\lambda_2(P_n) = \Theta\parentheses*{\frac{1}{n^2}}$.\footnote{Proof in \cref{lem:c1}.}
\end{lem}

\begin{lem}
$\lambda_n(P_n) \in [1,4]$.\footnote{Proof in \cref{lem:c2}.}
\end{lem}

\subsection{Complete Binary Tree}

\begin{lem}
$\lambda_2(T_d) = \Theta\parentheses*{\frac{1}{n}}$.\footnote{Proof in \cref{lem:c3}.}
\end{lem}

\begin{lem}
$\lambda_n(T_d) \in [1,6]$.\footnote{Proof in \cref{lem:c4}.}
\end{lem}