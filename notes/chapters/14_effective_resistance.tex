% !TeX root = ../main.tex
% Add the above to each chapter to make compiling the PDF easier in some editors.

\chapter{Effective Resistance}\label{cha:effective_resistance}

\begin{defn}[Effective resistance] The \emph{effective resistance}\index{effective resistance}, \begin{align}
    \Reff(a,b) \defeq \min_{\substack{\vf \in \R^{|\sE|} \\ \mB\vf = \vOne_b - \vOne_a}} \mathcal{E}(\vf),
\end{align} is the minimum electrical energy required to route one unit of flow from $a$ to $b$.
\end{defn}
\begin{rmk}
Per definition of electrical energy, routing $F$ units of flow from $a$ to $b$ costs $F^2 \Reff(a,b)$.
\end{rmk}

\begin{lem}
$\Reff(a,b) = \norm{\mL^{\nicefrac{+}{2}}(\vOne_b - \vOne_a)}_2^2$.
\end{lem}
\begin{proof} As the electrical flow $\vf$ is energy-minimizing, we have that $\Reff(a,b) = \trans{\vf}\mR\vf$. Recall that by Ohm's law this flow corresponds to voltages $\vx$ solving $\mL\vx = \vOne_b - \vOne_a$, that is, $\vx = \pinv{\mL}(\vOne_b - \vOne_a)$. We obtain, \begin{align*}
    \Reff(a,b) = \trans{\vf}\mR\vf = \trans{\vx}\mL\vx &= \trans{(\vOne_b - \vOne_a)}\pinv{\mL}\mL\pinv{\mL}(\vOne_b - \vOne_a) \\
    &= \trans{(\vOne_b - \vOne_a)}\pinv{\mL}(\vOne_b - \vOne_a) \margintag{using that $\vOne_b - \vOne_a \perp \vOne$} \\
    &= \norm{\mL^{\nicefrac{+}{2}}(\vOne_b - \vOne_a)}_2^2. \qedhere
\end{align*}
\end{proof}

\begin{lem}
$\E{C_{a,b}} = \norm{\vd}_1 \Reff(a,b)$.
\end{lem}
\begin{proof} Recall that $\E{C_{a,b}} = \trans{(\vOne_a - \vOne_b)}\vx$ for a solution $\vx$ to $\mL\vx = \norm{\vd}_1 (\vOne_a - \vOne_b)$, that is, $\vx = \norm{\vd}_1\pinv{\mL}(\vOne_a - \vOne_b)$. Now, observe that, \begin{align*}
    \Reff(b,a) = \trans{(\vOne_a - \vOne_b)}\pinv{\mL}(\vOne_a - \vOne_b) = \frac{1}{\norm{\vd}_1}\trans{(\vOne_a - \vOne_b)}\vx.
\end{align*} Thus, $\E{C_{a,b}} = \norm{\vd}_1 \Reff(b,a)$. Using symmetry of the commute time, $\E{C_{a,b}} = \E{C_{b,a}} = \norm{\vd}_1 \Reff(a,b)$.
\end{proof}

\begin{cor}\label{cor:effective_resistance_symmetric}
Effective resistance is symmetric.
\end{cor}

Let us consider a few examples.

\begin{marginfigure}
TBD
% \centering\includegraphics[width=4cm]{notes/figures/convex_set.png}
\caption{Sequential resistors.}\label{fig:sequential_resistors}
\end{marginfigure}
\begin{lem}
For the graph of \cref{fig:sequential_resistors}, $\Reff(1,k+1) = \sum_{i=1}^k \vr(i)$.
\end{lem}
\begin{proof}[Proof sketch] For the flow to be $1$, by Ohm's law, the voltage difference across edge $i$ must be $\vr(i)$.
\end{proof}

\begin{marginfigure}
TBD
% \centering\includegraphics[width=4cm]{notes/figures/convex_set.png}
\caption{Parallel resistors.}\label{fig:parallel_resistors}
\end{marginfigure}
\begin{lem}
For the graph of \cref{fig:parallel_resistors}, $\Reff(1,2) = \frac{1}{\sum_{i=1}^k \nicefrac{1}{\vr(i)}}$.
\end{lem}
\begin{proof}[Proof sketch] For the flow to be $1$, by Ohm's law, we must have, \begin{align*}
    1 = \sum_{i=1}^k \frac{\Delta}{\vr(i)},
\end{align*} where $\Delta$ is the voltage difference between vertices $1$ and $2$. Note that $\Reff(1,2) = \Delta$.
\end{proof}

\section{Effective Resistance as a Metric}

Before showing that effective resistance is a metric on the set of vertices, we consider the following lemma. We will write, \begin{align}
    \vx_{a,b} \defeq \pinv{\mL}(\vOne_b - \vOne_a),
\end{align} for the electrical voltages required to route one unit of current from $a$ to $b$.

\begin{lem}\label{lem:voltages_are_weighted_average}
If $\vx_{a,b}$ is a solution to $\mL\vx_{a,b} = \vOne_b - \vOne_a$, then we have for all $c \in \sV$ that $\vx_{a,b}(b) \geq \vx_{a,b}(c) \geq \vx_{a,b}(a)$.
\end{lem}
\begin{proof}[Proof sketch] Consider any $c \in \sV \setminus \{a,b\}$. Then, $(\mL\vx_{a,b})(c) = 0$. Thus, \begin{align*}
    \parentheses*{\sum_{v \sim c} \vw(\{v,c\})} \vx_{a,b}(c) - \parentheses*{\sum_{v \sim c} \vw(\{v,c\}) \vx_{a,b}(v)} = 0.
\end{align*} So, we have, \begin{align*}
    \vx_{a,b}(c) = \frac{\sum_{v \sim c} \vw(\{v,c\}) \vx_{a,b}(v)}{\sum_{v \sim c} \vw(\{v,c\})}.
\end{align*} In words, the voltage of $c$ is a weighted average of the voltages of its neighbors. It follows that the voltages of $a$ and $b$ take the largest absolute values.
\end{proof}

\begin{defn}[Metric] A \emph{metric}\index{metric} on a set $\sS$ is a function $d : \sS \times \sS \to \R$ such that for any $a, b, c \in \sS$, \begin{enumerate}
    \item $d(a,b) = 0 \iff a = b$;
    \item $d(a,b) \geq 0$;
    \item $d(a,b) = d(b,a)$; and
    \item $d(a,b) \leq d(a,c) + d(c,b)$.
\end{enumerate}
\end{defn}
\begin{lem}
Effective resistance is a metric on $\sV$.
\end{lem}
\begin{proof} It is easy to check that properties (1) and (2) are satisfied. We have that property (3) is satisfied by \cref{cor:effective_resistance_symmetric}.

Let us therefore consider property (4), the triangle inequality. We have, \begin{align*}
    \vx_{a,b} = \pinv{\mL}(\vOne_b - \vOne_a) = \pinv{\mL}(\vOne_c - \vOne_a + \vOne_b - \vOne_c) = \vx_{a,c} + \vx_{c,b},
\end{align*} This we can use to rephrase the effective resistance, \begin{align*}
    \Reff(a,b) = \trans{(\vOne_b - \vOne_a)}\vx_{a,b} &= \trans{(\vOne_b - \vOne_a)}(\vx_{a,c} + \vx_{c,b}) \\
    &= \vx_{a,c}(b) - \vx_{a,c}(a) + \vx_{c,b}(b) - \vx_{c,b}(a) \\
    &\leq \vx_{a,c}(c) - \vx_{a,c}(a) + \vx_{c,b}(b) - \vx_{c,b}(c) \margintag{using \cref{lem:voltages_are_weighted_average}} \\
    &= \trans{(\vOne_c - \vOne_a)}\vx_{a,c} + \trans{(\vOne_b - \vOne_c)}\vx_{c,b} \\
    &= \Reff(a,c) + \Reff(c,b). \qedhere
\end{align*}
\end{proof}