% !TeX root = ../main.tex
% Add the above to each chapter to make compiling the PDF easier in some editors.

\chapter{Spectral Graph Sparsification}

Many combinatorial graph algorithms perform better on sparse graphs. In this chapter, we will see that for any dense graph, we can find a sparse graph with approximately the same Laplacian matrix as measured by quadratic forms.

\begin{defn}[Matrix approximation]\index{matrix approximation} Given $\mA, \mB \in \sS_+^n$ and $\epsilon > 0$, we say, \begin{align}
    \mA \approx_\epsilon \mB \quad\text{iff}\quad \frac{1}{1+\epsilon}\mA \preceq \mB (1+\epsilon)\mA.
\end{align}
\end{defn}

Given some graph $G = (\sV,\sE,\vw)$, our goal is to find a graph $\Tilde{G} = (\sV,\Tilde{\sE},\Tilde{\vw})$ such that $|\Tilde{\sE}| \ll |\sE|$ and $\mL_G \approx_\epsilon \mL_{\Tilde{G}}$. We will write $\mL \defeq \mL_G$ and $\Tilde{\mL} \defeq \mL_{\Tilde{G}}$.

\begin{lem}
If $\mL \approx_\epsilon \Tilde{\mL}$, then for any cut $(\sS, \sV \setminus \sS)$, \begin{align}
    \frac{1}{1+\epsilon}c_G(\sS) \leq c_{\Tilde{G}}(\sS) \leq (1+\epsilon)c_G(\sS).
\end{align}
\end{lem}
\begin{proof}
Recall that $c_G(\sS) = \trans{\vOne_\sS}\mL\vOne_\sS$. The statement follows immediately by comparing the quadratic forms.
\end{proof}

\begin{thm}[Spectral graph approximation by sampling]
For any $\epsilon, \delta \in (0,1)$, there exist sampling probabilities $p_e$ for each edge $e \in \sE$ such that if $e \in \Tilde{\sE}$ with probability $p_e$ and $\Tilde{\vw}(e) \defeq \nicefrac{\vw(e)}{p_e}$, then with probability at least $1-\delta$ the graph $\Tilde{G} = (\sV,\Tilde{\sE},\Tilde{\vw})$ satisfies,\cite{spielman2011graph} \begin{align*}
    \mL \approx_\epsilon \Tilde{\mL} \quad\text{and}\quad |\Tilde{\sE}| = \LandauO{\frac{n}{\epsilon} \log\parentheses*{\frac{n}{\delta}}}.
\end{align*}
\end{thm}
\begin{proof}
TBD
\end{proof}